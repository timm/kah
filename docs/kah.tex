\documentclass[10pt,portrait]{book}
\usepackage[T1]{fontenc}
\usepackage{inconsolata}
\usepackage{textcomp}

\usepackage{xcolor}
\usepackage{graphicx}
\usepackage[margin=1cm]{geometry}

% \usemintedstyle{tango}
% \setminted[lua]{%
%   frame=lines,%
%   fontsize=\scriptsize,%
%   breaklines=true,%
%   mathescape=true,%
% }
%
\usepackage[cache=false]{minted}
\usepackage{unicode-math}
\usemintedstyle{tango}
\definecolor{bg}{rgb}{0.95, 0.95, 0.95} % very light gray
\definecolor{shadow}{rgb}{0.85, 0.85, 0.85} % slightly darker for shadow

\setminted[lua]{%
  frame=single,%
  framesep=2mm,%
  fontsize=\fontsize{6}{6}\selectfont,%
  baselinestretch=1.2,%
  breaklines=true,%
  mathescape=true,%
  bgcolor=bg,%
}

\usepackage{enumitem}
\setitemize{noitemsep,topsep=0pt,parsep=0pt,partopsep=0pt}

\definecolor{codegreen}{rgb}{0,0.6,0}
\definecolor{codegray}{rgb}{0.5,0.5,0.5}
\definecolor{codepurple}{rgb}{0.58,0,0.82}
\definecolor{backcolour}{rgb}{0.95,0.95,0.92}

\title{adasas}
\author{Tim Menzies}
\begin{document}
\maketitle
\small

\twocolumn
 This Lua script is a program designed for sequential
   model-based optimization using a Tree-structured Parzen Estimator
   (TPE) and a Bayesian classifier. It facilitates experimentation
   with various configuration parameters, making it suitable for tasks
   involving statistical confidence testing, bootstrap sampling, and
   classification.
   \begin{itemize}
   \item
  Users can specify a training data file and adjust several experiment
   settings to fit different data or optimization tasks. The script
   includes commands for displaying help, setting a random seed, and
   selecting data partitions for training and testing. 
   \item
  For a review of the code intelligence, look below for `acquire`.
  \item
  The code is divided into four sections:
   \begin{itemize}
   \item
      A help string, from which we parse out the settings (into a variable called `the`);
     \item Some general utility functions;
     \item Some classes;
     \item A library of start-up actions called `EG`;
     \item The actual start-up actions.
   \end{itemize}
   \end{itemize}

\begin{figure}
\inputminted{lua}{luas/Bayes.lua}
\end{figure}

\begin{figure}
\inputminted{lua}{luas/Data.lua}
\end{figure}
\begin{figure}
\inputminted{lua}{luas/Dist.lua}
\end{figure}
\begin{figure}
\inputminted{lua}{luas/EG.lua}
\end{figure}
\begin{figure}
\inputminted{lua}{luas/Start-up.lua}
\end{figure}
\begin{figure}
\inputminted{lua}{luas/Starting.lua}
\end{figure}
\begin{figure}
\inputminted{lua}{luas/Stats.lua}
\end{figure}
\begin{figure}
\inputminted{lua}{luas/Utilities.lua}
\end{figure}



\end{document}
